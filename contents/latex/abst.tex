\documentclass[9pt,a4j]{jsarticle} %9pt,A4判
\usepackage{amsmath}
\usepackage{amssymb}
\usepackage{amsthm}
\usepackage{ascmac}
\usepackage[dvipdfmx]{graphicx}
\usepackage{here}
\usepackage{fancybox}
\usepackage{cases}
\usepackage{bm}
\usepackage{array}
\usepackage{comment}
\usepackage{okumacro} %ルビの挿入
%
%--余白等の調整--
\usepackage[]{multicol}
\usepackage[top=20mm,bottom=20mm,left=23mm,right=23mm]{geometry}%全ページの余白の定義
\usepackage{titlesec}%セクション等大きさパッケージ
%\newcommand{\beforesection}[1]{\vspace{-#1mm}}%section前余白設定用タグの宣言 (VV)
\pagestyle{empty} %頁番号の消去
%
%---------------------
\newtheorem{thm}{定理}[subsection]
\newtheorem{prop}[thm]{命題}
\newtheorem{lemma}[thm]{補題}
\newtheorem{exercise}{演習問題}
\newtheorem{example}{例}
\newtheorem*{example*}{例}
\newtheorem{lemm}{補題}
\newtheorem{cor}{系}
\newtheorem{defi}[subsubsection]{定義}
\newtheorem*{defi*}{定義}
\renewcommand\proofname{\textbf{証明}}
%
%--Dirac表記のコマンド等---
\newcommand{\ket}[1]{| #1 \rangle}
\newcommand{\bra}[1]{\langle #1 |}
\newcommand{\norm}[1]{|| #1 ||}
\newcommand{\bk}[2]{ \langle#1|#2\rangle }
\newcommand{\kb}[2]{ |#1\rangle\langle#2| }
\newcommand{\Hi}{\mathcal{H}}
\newcommand{\I}{\mathbb{I}}
\newcommand{\R}{\mathbb{R}}
\newcommand{\E}{\mathbb{E}}
%
%--式番号を章毎にする--
%
\makeatletter
\@addtoreset{equation}{section}
\def\theequation{\thesection.\arabic{equation}}
\makeatother
%%
\renewcommand{\abstractname}{} %abstractのタイトルを消す
\begin{document}
%
%%--タイトル・著者--
\begin{flushright}
2021年2月6日
\end{flushright}
\begin{center}
\LARGE{タイトル}\\
\large{English Title}
\end{center}
\noindent
システム理工学専攻 \hfill MF19000 \ \ruby{苗字}{みょうじ}\ \ruby{名前}{なまえ} \\
量子情報システム研究 \hfill 指導教員 \ 木村\ 元\ \ \ \ 
%
%--abstruct-----------------------
\begin{abstract}
area for writing abstract.
\end{abstract}
%
%--本文--------------------------
%\beforesection{1}%セクション前余白縮小 (VV)
\begin{multicols}{2}%2段組み
\section{セクション}
本文。変更の可能性があるので、フォーマット指定は必ず確認すること。\\
2段組み\\
余白。上下20mm,左右23mm。\\
文字9pt、目安は9~10.5pt\\
フォント埋め込み\\
ページ番号は入れない

%
%--引用--------------------------
%\begin{thebibliography}{20}
% \bibitem{}
%\end{thebibliography}
%
\end{multicols}
\end{document}
